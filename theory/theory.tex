\documentclass[12pt]{article}

% Encoding and font packages
\usepackage[utf8]{inputenc}
\usepackage[T1]{fontenc}
\usepackage{lmodern}  % Improved font quality

% Useful packages
\usepackage{amsmath, amssymb} % For math symbols and equations
\usepackage{graphicx}         % For including images
\usepackage{hyperref}         % For clickable links in the PDF
\usepackage{lipsum}           % For dummy text (lorem ipsum)

% Title, Author, and Date
\title{Trafing strategies theory}
\author{Andrei Nekliudov}
\date{\today}

\begin{document}
	
	\maketitle
	
	\section{Financial Forecasting And Trading Strategies}
	\subsection{MA}
		Moving Average rules (MAs) are also common mechanical indicators and their applications are
		known for many decades in trading decisions and systems. In simple words, a MA is the mean of a
		time series, which is recalculated every trading day. Their main characteristic is the length window,
		namely the number of trading days that are going to be used to calculate the rolling mean of the
		high frequency data. MAs are identifiers of short- or long-term trends, so the window length can be
		short (short MAs – 1 to 5 lags) or long (long MAs- 10 to 100 lags). The intuition behind them isthat buy (sell) signals are triggered when closing prices cross above (below) the x day MA. Another
		variation is to buy (sell) when x day MA crosses above (below) the y day MA.
		
		Assuming that the length window is n days, the current period’s t closing price $P_t$, MAs can be
		further divided into three main categories
		
		\begin{itemize}
			\item Simple MA (SMA): $SMA_{t+1} = \frac{P_t + P_{t-1} + ... + P_{t-n + 1}}{n}$
			
			\item Exponential MA(EMA): $EMA_{t+1} = EMA_t + \alpha(P_t - EMA_t)$
			
			\item Weighted MA (WMA): $WMA_{t + 1} = \frac{nP_t + (n-1)P_{t - 1} + ... + 2P_{t - n + 2} + P_{t-n+1}}{n(n + 1) / 2}$
		\end{itemize}
	
	The SMA is an average of values recalculated every day. The EMA is adapting to the market price
	changes by smoothing constant parameter α. The smoothing parameter expresses how quickly the
	EMA reacts to price changes. If α is low, then there is little reaction to price differences and vice
	versa. The WMA give weights to the prices used a lags. These weights are higher in recent periods,
	giving higher importance in recent closing prices. All these MAs are using the closing price as the
	calculation parameter, but open, high and low prices could also be used.
	\subsection{Oscillators(OTs) and Momentum Rules(MTs)}
	The third class of mechanical trading rules consists of the Oscillators (OTs) and Momentum Rules
	(MTs). OTs are techniques that do not follow the trend. Actually, they try to identify when the trend
	is apparent for too long or ‘dying’. Therefore they are also called ‘non-trending market indicators’.
	The main drawback of MAs is the inability to identify the quick and violent swifts in price
	direction, which lead to capital loss by generating wrong trading signals. This performance gap is
	filled from OT indices. Their basic intuition is that a reversal trend is eminent, when the prices
	move away from the average. Simple OT rules are based on the difference between two MAs and
	generate buy (sell) signals when prices are too low (have risen extremely). Nonetheless, being a
	difference of MA rules, OTs can also present buy and sell position, when the index crosses zero.
	The boundaries between OTs and MTs can be a bit vague depending on the case, because MTs can
	be applied to MAs and OTs. The main difference is that OTs are non-trend indicators, whereas MTs
	are capitalizing on the endurance of a trend in the market. A simple MT rule would be the
	difference between today’s closing price and the closing price of x days ago. The trading signal is
	generated based on this momentum. To put it simply, the buy (sell) signal is given when today’s
	closing price is higher than the closing price x days ago. Setting properly the x day’s price that is
	going to be used is also a matter of trader intuition, market knowledge and historical experience (5
	and 20 days are common).There are many types of OTs and MTs used in trading applications. Some typical examples are
	summarized, interpreted in short and followed by relevant research applications below:
	
	\begin{itemize}
		\item Moving Average Convergence/Divergence (MACD):
		MACD is calculated as the
		difference between short- and long-term EMAs and identifies where crossovers and
		diverging trends to generate buy and sell signals.
		\item Accumulation/Distribution (A/D): A/D is a momentum indicator which measures if
		investors are generally buying (accumulation) or selling (distribution) base on the volume
		of price movement.
		\item Chaikin Oscillator (CHO): CHO is calculated as the MACD of A/D.
		
		\item Relative Strength Index ( SI): The SI is calculated based on the average ‘up’ moves and
		average ‘down’ moves and is used to identify overbought (when its value is over 70 – sell
		signal) or oversold (when its value is under 30-buy)
		
		\item Price Oscillator (PO): PO is identifying the momentum between two EMAs.
		
		\item Detrended Price Oscillator (DPO): DPO eliminates long-term trends in order to easier
		identify cycles and measures the difference between closing price and SMA.
		
		\item Bollinger bands (BB): BB are based on the difference of closing prices and SMAs and
		determine if securities are overbought or oversold.
		
		\item Stochastic Oscillator (SO): SO is based on the assumptions that as prices rise, the closing
		price tends to reach the high prices in the previous period.
		
		\item Triple EMA (TRIX): TRIX is a momentum indicator between three EMAs and triggers
		buying and selling signals base on zero crossovers.
	\end{itemize}
	The exact specifications and formulas of the abovementioned indicators can be found in Gifford
	(1995), Chang et al. (1996) and Edwards and Magee (1997) or in any common textbook of
	technical analysis. Their utility though has been eminent years before that. The pioneering paper of
	Brock et al. (1992) presents evidence of profitability of MACD, as for MAs and FRs mentioned
	above. Kim and Han (2000) propose a hybrid genetic algorithm – neural network model that uses
	OTs, such as PO, SO, A/D and RSI, along with simple momentum rules to predict the stock market.
	Leung and Chong (2003) compare the profitability of MA envelopes and BBs. Their results suggest
	BBs do not outperform the MA envelopes, despite being able to capture sudden price fluctuations.
	Shen and Loh (2004) propose a trading system with rough sets to forecast S&P 500 index, which
	outperforms BH rules. In order to set up this hybrid trading system, they search for the most
	efficient rules based on the historical data from a pool of technical indicator, such as MACD, RSI
	and SO. Lento et al. (2007) also present empirical evidence that prove BBs’ inability to achieve
	higher profits compared to a BH strategy, when tested on tested on the S&P/TSX 300 Index, the
	Dow Jones Industrial Average Index, NASDAQ Composite Index and the Canada/USD exchange
	rate. Chong and Ng (2008) examine the profitability of MACD and RSI using 60-year data of the
	London Stock Exchange and found that the RSI as well as the MACD rules can generate returns
	higher than the BH strategy in most cases.


	
	\section{Introduction}
	This document serves as a sample for a \LaTeX\ file in TeXstudio. LaTeX is a powerful typesetting system widely used for creating scientific documents, articles, reports, and more.
	
	\section{Basic Elements}
	Below are some common elements used in a LaTeX document.
	
	\subsection{Text and Paragraphs}
	LaTeX automatically handles spacing between paragraphs. For example:
	
	\lipsum[1] % Generates a paragraph of Lorem Ipsum text.
	
	\subsection{Mathematical Equations}
	LaTeX excels at typesetting mathematics. Here is an example of an inline equation: \( E = mc^2 \). And here is a displayed equation:
	
	\[
	\int_{a}^{b} f(x) \, dx = F(b) - F(a)
	\]
	
	\subsection{Including Graphics}
	You can include images using the \texttt{graphicx} package. For example:
	
	\begin{figure}[h!]
		\centering
		\includegraphics[width=0.5\textwidth]{example-image} % Ensure you have a file named "example-image.pdf/jpg/png" in your project directory
		\caption{Sample Image}
		\label{fig:sample}
	\end{figure}
	
	\subsection{Hyperlinks}
	The \texttt{hyperref} package makes it easy to add hyperlinks. For example, visit the \href{https://www.latex-project.org/}{LaTeX Project website}.
	
	\section{Conclusion}
	This sample document provides a basic framework for creating documents in LaTeX with TeXstudio. From here, you can expand your document with more sections, figures, tables, and custom formatting as needed.
	
\end{document}
